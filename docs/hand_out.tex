\documentclass[10pt]{article}
 \pagestyle{empty}
 \usepackage{color}
  \definecolor{gray}{gray}{0.4}
  \definecolor{bg}{gray}{0.9}
 \usepackage{graphics}
 \parindent        0 mm
 \topmargin      -32 mm
 \oddsidemargin   -7 mm
 \textheight     284 mm
 \textwidth      180 mm
 \unitlength       1 mm
 \newcommand{\sclc}[1]{\newline\texttt{\color{blue}\phantom{xxx}#1}\newline }
                %
                \newcommand{\ts}[1]{\texttt{\textsl{#1}}}
                \newcommand{\<}{$<$}
                \renewcommand{\>}{\/$>$}
                %
                \newcommand{\sics}[1]{\mbox{\texttt{%
                 {\color{gray} \rule[-0.3ex]{6ex}{1.9ex}}\hspace{-5.4ex}%
                 {\color{white}SICS}\hspace{1ex} #1 }}%
                }

\begin{document}
\sf
\parbox[t]{80mm}{\textbf{\large short users' guide for \\[2ex] 
                          \Huge Amor}} \hfill
\parbox[t]{55mm}{
\begin{tabular}[t]{@{}lr}
 Amor cabin   & 056 310 3593 \\
 Amor area    & '' 4605 \\
 Jochen Stahn & '' 2518 \\
 Thomas Geue  & '' 5988 \\
 stand-by service & (0)079 820 2160
\end{tabular}}%

\rule{\textwidth}{0.8mm}

\vfill
\textbf{alignment:}\\[1mm]
for alignment {\color{red}light} following the neutron path can be activated
as alternative to the frame-overlap filter:
 \sclc{conf light|filter}
for the real measurements it is \textbf{essential} to return to the 
configuration \textsl{filter} \\

there are 3 principle set-ups: horizontal, deflected unpolarised, or 
 deflected polarised;
these are accessible via: 
 \sclc{conf horizontal | polariser | deflector [<value>]}

you can switch between single and area detector with:
 \sclc{conf single|area}



\vfill
\textbf{important commands:}
\begin{picture}(0,0)
 \put(24,-43){
  \begin{picture}(80,50)
    \put(-01,039){\color{bg}\rule{14mm}{8mm}}
    \put(059,039){\color{bg}\rule{17mm}{12mm}}
    \put(077,039){\color{bg}\rule{16mm}{8mm}}
    \put(090,023){\color{bg}\rule{12mm}{14mm}}
   \thinlines
    \put(092,025){\color{gray}\line(0,1){10}}
    \put(100,025){\color{gray}\line(0,1){10}}
    \put(000,046){\color{blue}\line(0,-1){11}}
    \put(003,042){\color{blue}\line(0,-1){12}}
    \put(030,050){\color{blue}\line(0,-1){34}}
    \put(060,050){\color{blue}\line(0,-1){19}}
    \put(078,046){\color{blue}\line(0,-1){15}}
    \put(085,042){\color{blue}\line(0,-1){11}}
    \put(030,008){\color{blue}\vector(0,1){3}}
    \put(030,004){\color{blue}\vector(0,-1){3}}
    \put(030,002){\color{blue}\vector(0,1){3}}
    \put(030,010){\color{blue}\vector(0,-1){3}}
    \put(030,015){\color{blue}\qbezier(9,-0)(9,2)(8,4)}
    \put(030,015){\color{blue}\qbezier(-8,-4)(-10,0)(-8,4)}
    \put(030,015){\color{blue}\line(1,0){20}}
    \put(028,006){\color{blue}\line(1,0){14}}
    \put(040,006){\color{blue}\line(0,1){9}}
   \thicklines
    \put(110,030){\line(-1,0){50}}
    \put(103,028){\color{gray}\line(0,1){4}}
    \put(106,031){\color{gray}\line(1,0){4}}
    \put(106,029){\color{gray}\line(1,0){4}}
    \put(060,030){\qbezier(0,0)(-4,-2)(-46,-23)}
    \put(030,015){\qbezier(0,0)(-4,2)(-29,14.5)}
   \put(000,048){\small detector}
    \put(001,044){\small area}
    \put(004,040){\small single}
    \put(000,024){\color{cyan}\rule{1mm}{10mm}}
    \put(002,027.7){\color{cyan}$\bullet$}
   \put(031,048){\small sample}
    \put(028,014.3){\color{red}\rule{4mm}{0.7mm}}
    \put(031,008){\small STZ}
    \put(041,010){\small ideal: 360\,mm}
    \put(031,002){\small SOZ}
    \put(040,016){\small $\color{red}\omega$=SOM}
    \put(008,014){\small $\color{red}2\theta$=S2T}
   \put(061,040){\small deflector}
    \put(061,044){\small polariser}
    \put(061,048){\small horizontal}
    \put( 56,029.2){\rotatebox{14}{\color{gray}\rule{8mm}{0.7mm}}}
   \put( 79,044){\small filter}
    \put( 86,040){\small light}
    \put( 70,029){\rotatebox{9}{\color{gray}\rule{8mm}{0.7mm}}}
    \put(085,025.5){\color{red}\line(0,1){4.5}}
    \put(081,018){\small \color{red}light}
    \put(090,020){\small \color{gray}chopper}
    \put(085,020){\color{red}\vector(0,1){5}}
    \put(083,032){\color{gray}\line(1,-1){4}}
    \put(006,030.7){\color{magenta}\small 5}
     \put(006,027.7){\color{magenta}\rule{1mm}{2mm}}
     \put(006,023.8){\color{magenta}\rule{1mm}{2mm}}
    \put(023,022){\color{magenta}\small 4}
     \put(023,019){\color{magenta}\rule{1mm}{2mm}}
     \put(023,015.7){\color{magenta}\rule{1mm}{2mm}}
    \put(036,022){\color{magenta}\small 3}
     \put(036,019){\color{magenta}\rule{1mm}{2mm}}
     \put(036,015.7){\color{magenta}\rule{1mm}{2mm}}
    \put(046.3,030.2){\color{magenta}\small slit 2}
     \put(052,027.2){\color{magenta}\rule{1mm}{2mm}}
     \put(052,023){\color{magenta}\rule{1mm}{2mm}}
    \put( 80.6,033.7){\color{magenta}\small 1}
     \put( 81,030.7){\color{magenta}\rule{1mm}{2mm}}
     \put( 81,027.5){\color{magenta}\rule{1mm}{2mm}}
  \end{picture}
 }
\end{picture}


check {\color{red}position} of \textsl{device}: 
 \sclc{<device>} 

{\color{red}drive} \textsl{device} to \textsl{value}: 
 \sclc{dr <device> <value>} 
 
{\color{red}redefine} the \textsl{value} for \textsl{device}: 
 \sclc{sp <device> <value>} 

to drive the vertical {\color{red}slit} \textsl{number} to \textsl{opening}: 
 \sclc{slit <number> <opening>} 
or, if all 5 slits should be changed: 
 \sclc{slit <opening1> <opening2> <opening3> <opening4> <opening5>} 
the present openings can be obtained by the command 
 \sclc{slit}  

\vfill
\textbf{scan commands:}\\[1mm]
{\color{red}center-scan} for \textsl{device} around \textsl{value} with \textsl{number}
steps to either side of \textsl{width} with a monitor \textsl{pre-set}: 
 \sclc{cscan <device> <center> <step-width> <number> <pre-set>} 

{\color{red}step-scan} for \textsl{device} (and \textsl{device'}) from \textsl{value1} to \textsl{value2}
with \textsl{number} points and monitor \textsl{pre-set}: 
 \sclc{sscan <device> <value1> <value2> [<device'> <value1'> <value2'>] <number> <pre-set>}

after a scan it is possible (please check the value reported back!)
to get a peak maximum with
 \sclc{peak} 
and then to drive there with 
 \sclc{center} 

\vfill
\textbf{magnets:}\\[1mm]
\parbox[t]{110mm}{
switch the {\color{red}polariser} [{\color{red}analyser}] to spin up
or down neutrons, or off:
 \sclc{spin +|0|- [+|0|-]}
    
drive the {\color{red}magnetic field} at the sample:
 \sclc{dr hsy <field in Oe>}
}   
\parbox[t]{65mm}{%
(re)mount polariser / analyser magnets:
 \sclc{conf pby|aby}
   
mount / unmount 1\,T electro magnet:
 \sclc{fmaset on|off}
}  \\ 


\vfill
\textbf{TOF data acqusition:}\\[1mm]
start a TOF {\color{red}measurement} for a monitor \textsl{pre-set}:
 \sclc{count monitor <pre-set>}  
the data file is created after finishing the data acquisition 
(a \textsl{pre-set} of 1e7 corresponds to about 35\,min) \\

a (simple) batch-job is a line-by-line list of the commands to be executed \\
define a \textsl{directory} where the \textsl{batch-file}s are located:
 \sclc{batchroot /home/amorlnsg/<your directory[/subdirectory]>}
execute the \textsl{batch-job}:
 \sclc{batchrun <batch-file>}
\\[-6ex]

\hfill{\color{gray}\tiny J.\ Stahn, \today}\\[-1.5ex]
\rule{\textwidth}{0.8mm}

%%%%%%%%%%%%%%%%%%%%%%%%%%%%%%%%%%%%%%%%%%%%%%%5
\clearpage
\textbf{The Chopper}\\[1ex]

In the standard configuration the {\color{red}ideal chopper speed is 1500\,rpm}.
Higher frequences will lead to an overlap of the pulses (unless the
chopper-detector distance and the frame overlap mirror are adjusted);
lower frequences are just a waste of neutrons. \\

To get the resolution $\Delta q_z/q_z$ constant it is essential to 
operate the chopper discs at a {\color{red}phase of $-13.6^\circ$}. Since there is
an offset at the second chopper disc, one has to give a value of
$-12.8^\circ$ (2013).  \\
Whenever the chopper is started it has problems to synchronise at
a phase other than 0. This means you first should go to the wanted
spped with phase 0 and then set the correct phase. \\

Sometimes (not very often!) there is a problem with the chopper. 
What can be wrong:
           \begin{itemize}
 \item The slave chopper disk lost its phase, master is still running: \\
       drive the speed to 0 by: \\
       press "F1" to access the master disk; \\
       press "R" and enter 0; \dots
 \item The cooling water failed (the red light at the cooling water 
       controll pannel is on): \\
       quit this state by pressing the red knop; \\
       restart the cooling water pump with the green knop; \dots
 \item The chopper controll computer has a problem: \\
       switch it off and on (its a DOS computer!); \\
       when booted enter \texttt{ncs} and wait; 
       no callipration here!; \dots
 \item[\dots] 
       switch the key on the chopper controll pannel to "hand"; \\
       press the button "reset"; \\
       press the button "callibrate"; \\
       switch the key back to "rechner";  \\
          \end{itemize}
Now the complete chopper system is ready to be started. To bring it to the
right parameters perform the following stepps:
          \begin{itemize}
 \item press "F2" to access the slave disk;
 \item press "P" and enter the phase 0;
 \item press "F1" to access the master disk;
 \item press "R" and enter the frequence (1500);
       wait until this is reached;
 \item press "F2" to access the slave disk;
 \item press "P" and enter the phase;
 \item check if the parameters are stable after some minute,
       otherwise reset the phase or restart the complete 
       procedure.
           \end{itemize}
The computer is really old and the software seems to be even older --- 
so be patient. It takes several seconds before a single input is
processed, even if it is only the change of controll with "F2" or "P". \\

There is a controll pannel in the left rack close to the area entrance.
If the chopper has a problem it will (most likely) be displayed there.
So please have a look there first to find out the origin of a problem.


%%%%%%%%%%%%%%%%%%%%%%%%%%%%%%%%%%%%%%%%%%%%%%%5
\clearpage
\textbf{Direct access to the motor controller} \\[1ex]
 The sics commands \sics{mot\ts{\<x\>}} give the opportunity to check or 
 change the parameters of the motor controllers.
 It is also possible, to run a motor or to perform reference runs.\\

 By using these commands you can destroy a lot! \\
 So, please {\color{red}use it only when you are sure what to do}! \\
 Other ways ask the instrument responsible! \\

 Syntax: \\[1ex]
  \sics{mot\ts{\<x\>} 
        send \ts{\<key\>} 
             \ts{\<slot\>} 
             [\ts{\<val1\>} [\ts{\<val2\>}]]} \\[1ex]
 with 
  \begin{tabular}[t]{lll}
   \ts{x}    & \textsf{A}, \textsf{B} or \textsf{C} 
          the name of the controller &
          table below \\
   \ts{key}  & the parameter to be set/read or&
          table next page \\ &
          the action to be performed &                           \\
   \ts{slot} & the slot-number of the device at 
          controller \ts{x} & 
          table below \\
   \ts{val1} & the first value for parameter/action 
          \ts{key} &
          table next page  \\
   \ts{val2} & the second value for parameter/action 
          \ts{key} &
          table next page  
  \end{tabular}\\[1ex]

 Example: Drive $\omega$ (device {\sf SOM}: controller \textsf{A}, slot 9) to $3.5^\circ$ \\[1ex]
  \sics{mota send p 9 3.5} 


 \vfill
\begin{tabular}{c|rc|ll}
controller & slot & device & group & motion  \\
\hline \hline
            &  1 &\sf COM & counter        & tilt \\
 \textsf{A} &  2 &\sf COZ &                & $z$ translation \\
            &  3 &\sf C3Z &                & $z$ position single counter\\
            &  4 &\sf COX &                & $x$ translation \\
            \cline{2-5}
            &  5 &\sf EOZ &                & $z$ lift Selene support\\
            &  6 &\sf XLZ &                & $z$ position distance laser\\
            &  7 &\sf EOM &                & $\omega$ Selene support\\
            &  8 &\sf --- &                & \\
            \cline{2-5}
            &  9 &\sf SOM & sample         & $\omega$ \\
            & 10 &\sf SOZ &                & $z$ lift of base \\
            & 11 &\sf STZ &                & $z$ translation on sample table \\
            & 12 &\sf SCH &                & $\chi$ \\
\hline \hline
            &  1 &\sf AOM & analyser       & tilt \\
 \textsf{B} &  2 &\sf AOZ &                & $z$ position of rotation axis \\
            &  3 &\sf ATZ &                & $z$ position relative to rotation axis \\
            &  4 &\sf --- &                & \\
            \cline{2-5}
            &  5 &\sf MOM & monochromator/ & tilt \\
            &  6 &\sf MOZ & polariser      & $z$ position of rotation axis \\
            &  7 &\sf MTZ &                & $z$ position relative to rotation axis \\
            &  8 &\sf MTY &                & $y$ position \\
            \cline{2-5}
            &  9 &\sf FOM & frame-overlap  & tilt \\
            & 10 &\sf --- &                & \\
            & 11 &\sf FTZ &                & $z$ position relative to rotation axis \\
            \cline{2-5}
            & 12 &\sf     &                &              \\
\hline \hline
            &  1 &\sf D5V & diaphragms     & 5 vertical \\
 \textsf{C} &  2 &\sf D5H &                & 5 horizontal \\
            &  3 &\sf D1L &                & 1 left \\
            &  4 &\sf D1R &                & 1 right \\
            &  5 &\sf D3T &                & 3 opening \\
            &  6 &\sf D3B &                & 3 $z$ position (lower edge) \\
            &  7 &\sf D4T &                & 4 opening \\
            &  8 &\sf D4B &                & 4 $z$ position (lower edge) \\
            &  9 &\sf D1T &                & 1 opening \\
            & 10 &\sf D1B &                & 1 $z$ position (lower edge) \\
            & 11 &\sf D2T &                & 2 opening \\
            & 12 &\sf D2B &                & 2 $z$ position (lower edge) \\
\hline \hline
 phytron    &  1 &\sf DST & luminous field & $y$ = transversal \\
            &  2 &\sf DSL & diaphragm      & $x$ = longitudinal \\
\hline \hline
\end{tabular}

           \clearpage
           \begin{table}[b] 
             \begin{tabular}{l|lcl|lll}
              \hline
  &\ts{key}    &\ts{slot}& value           & type& unit   & meaning                           \\
              \hline
  &\texttt{mn} &\ts{m}&[\ts{val1}]     & text&        & motor name                        \\
  &\texttt{ec} &\ts{m}&[\ts{val1 val2}]& int &        & encoder mapping (type/number)     \\
  &\texttt{ep} &\ts{m}&[\ts{val1}]     & int &        & encoder magic parameter           \\
  &\texttt{a}  &\ts{m}&[\ts{val1}]     & int & digits & precision                         \\
  &\texttt{fd} &\ts{m}&[\ts{val1 val2}]& int &        & encoder gearing (number/denom)     \\
  &\texttt{fm} &\ts{m}&[\ts{val1 val2}]& int &        & motor   gearing (number/denom)     \\
  &\texttt{d}  &\ts{m}&[\ts{val1}]     & real& deg?   & inertia tolerance                 \\
  &\texttt{e}  &\ts{m}&[\ts{val1 val2}]& int & kHz/s  & start/stop ramp          \\
  &\texttt{f}  &\ts{m}&[\ts{val1}]     & int &        & open loop/closed loop (0/1)       \\
  &\texttt{g}  &\ts{m}&[\ts{val1}]     & int & ms/s   & start/stop frequency  \\
  &\texttt{h}  &\ts{m}&[\ts{val1 val2}]& real& deg/mm & low/high software limits          \\
    \rotatebox{90}{\makebox[1ex]{parameter}}
  &\texttt{j}  &\ts{m}&[\ts{val1}]     & int & ms/s   & top speed                         \\
  &\texttt{k}  &\ts{m}&[\ts{val1}]     & int &        & reference mode :                  \\
  &            &      &                &     &        &--11  low limit + index is reference point.\\
  &            &      &                &     &        & --1  low limit is reference point.        \\
  &            &      &                &     &        &   0  absolute encoder              \\
  &            &      &                &     &        &   1  high limit is reference point.        \\
  &            &      &                &     &        &   2  separate reference point.        \\
  &            &      &                &     &        &  11  high limit + index is reference point.\\
  &            &      &                &     &        &  12  separate + index is reference point.\\
  &\texttt{l}  &\ts{m}&[\ts{val1}]     & int & ms     & backlash                          \\
  &\texttt{m}  &\ts{m}&[\ts{val1}]     & int & es     & position tolerance                \\
  &\texttt{q}  &\ts{m}&[\ts{val1}]     & ?   &        & reference switch width            \\
  &\texttt{t}  &\ts{m}&[\ts{val1}]     & int &        & one-sided operation flag (0 = no) \\
  &\texttt{v}  &\ts{m}&[\ts{val1}]     & int &        & null point                        \\
  &\texttt{w}  &\ts{m}&[\ts{val1}]     & int &        & air cushion dependency            \\
  &\texttt{z}  &\ts{m}&[\ts{val1}]     & int & es     & circumf. of encoder               \\
              \hline
             %\end{tabular} 
           %\end{table}
           %
           %\begin{table}[b] 
           % \caption{ \label{controlleract}
           %Commands for the step-motor controller EL734.
           % }
           % \begin{tabular}{lcl|lll}
           %\hline
           %\ts{key} & slot & value & type & unit & meaning                           \\
           %\hline
  &\texttt{am} &        &           & hex & \#   & active motors, status                          \\
  &\texttt{u}  & \ts{m} &[\ts{val1}]& real& deg  & set actual value (\texttt{v} recalculated)  \\
  &\texttt{uu} & \ts{m} &           & real& deg  & set actual value (\texttt{v} unchanged)     \\
  &\texttt{ud} & \ts{m} &           & hex & ds   & read actual value                           \\
  &\texttt{p}  & \ts{m} &           & real& deg  & read last target value                     \\
  &\texttt{p}  & \ts{m} &[\ts{val1}]& real& deg  & drive motor to target value                \\
  &\texttt{pd} & \ts{m} &[\ts{val1}]& hex & ds   & drive motor to target value                \\
  &\texttt{pr} & \ts{m} &[\ts{val1}]& int & ms   & drive motor d steps                    \\
  &\texttt{n}  & \ts{m} &           &     &      & drive free from end switch             \\
  &\texttt{s}  &[\ts{m}]&           &     &      & stop [all] motor[s]                    \\
  &\texttt{r}  & \ts{m} &           &     &      & reference run (\texttt{u} changed)     \\
  &\texttt{rf} & \ts{m} &           &     &      & reference run (\texttt{u} unchanged)   \\
  &\texttt{ff} & \ts{m} &[\ts{val1}]& int & ms/s & drive forward with ramp and frequency \ts{val1}   \\
  &\texttt{fb} & \ts{m} &[\ts{val1}]& int & ms/s & drive backward with ramp and frequency \ts{val1}  \\
  &\texttt{sf} & \ts{m} &[\ts{val1}]& int & ms/s & drive forward with frequency f            \\
  &\texttt{sb} & \ts{m} &[\ts{val1}]& int & ms/s & drive backward with frequency f           \\
  &\texttt{ac} & \ts{m} &           & int & \#   & status air cushions (on/of: 1/0)       \\
    \rotatebox{90}{\makebox[1ex]{commands}}
  &\texttt{ac} & \ts{m} &[\ts{val1}]& int & \#   & de-/activate air cushions (0/1)        \\
  &\texttt{so} & \ts{m} &[\ts{val1}]& int & \#   & set output high/low (1/0)              \\
  &\texttt{ri} &        &           & bin & \#   & read all inputs                                \\
  &\texttt{ri} & \ts{m} &           & int & \#   & read input m                           \\
  &\texttt{msr}& \ts{m} &           & hex & \#   & run status m                            \\
  &\texttt{ss} & \ts{m} &           & hex & \#   & show status flags ...                  \\
  &\texttt{de} & \ts{m} &           &     &      & set default all parameters             \\
  &\texttt{echo}&       &[\ts{val1}]& int & \#   & 0 host-mode without echo                      \\
  &            &        &           &     &      & 1 terminal-mode with echo                     \\
  &            &        &           &     &      & 2 host-mode, messages delayed                 \\
  &\texttt{rmt}&        & \ts{val1} & int & \#   & set remote flag (1/0)                          \\
  &\texttt{rmt}&        &           & int & \#   & read remote (0: offline, 1: host-port active)   \\
  &\texttt{\%} &        &           &     &      & stop all motors, reset with default parameters \\
  &\texttt{?}  &        &           &     &      & help                                           \\
  &\texttt{?c} &        &           &     &      & command list                                   \\
  &\texttt{?m} &        &           &     &      & message list                                   \\
  &\texttt{?p} &        &           &     &      & parameter list                                 \\
                 \hline
 \multicolumn{7}{l}{deg=degree, ms=motor steps, ms/s=motor steps per second, es=encoder steps,
    ds= ? steps}\\
            \end{tabular}
           \end{table}

%======================================================================
\clearpage

\texttt{amorreducer} \\

The perl program \texttt{amorreducer} can be used to extract data 
stored in \texttt{.hdf} format and to correct (better \textsl{normalyse})
them to get real reflectivity curves $R(q_z,q_x)$ instead of 
intensity vs. time $I(t,z)$, where $t$ is the flight time from the
chopper to the detector.  
$z$ is the vertical position on the area detector.
The raw data $I(t,z)$ are not corrected for the
wavelength dependence of the incomming beam intensity $I_0(\lambda)$.  \\

Conversion of $I(t,z)$ to $R(q_z,q_x)$:  \\
\begin{tabular}{llp{120mm}}
1.&normalisation:
  &$R(t,z) = \mbox{scale} \cdot I(t,z)/I_0(t)$ \\
2.&convert $t$ to $\lambda$:
  &$\lambda = ??? \cdot t / \mbox{chopper-to-detector-distance}$ \\
3.&convert $\lambda$ and $z$ to $q_z$ and $q_x$: 
  &$q_z = (\sin \alpha_i + \sin \alpha_f)\,2\pi/\lambda $ \\ &
  &$q_x = (\cos \alpha_i - \cos \alpha_f)\,2\pi/\lambda$ \\ 
  &&{with the angle of incidence $\alpha_i$ and
   the final angle $\alpha_f$;
   for the specular case $\alpha_f=\alpha_i=\theta$, $q_x=0$ 
   (mind the different coordinate systems!).} 
\end{tabular} \\

The scale factor has to be taken from theory (if the sample is much
larger than the beam or is completly bathed) including diaphragm opeings
OR it has to be adjusted by a sophisticated method (i.e. by eye). \\

usage: \\
\texttt{amorreducer} <file> \\
where file contains the order to \texttt{amorreducer} what to do
with the according parameters. A typical parameter file looks like
this:

...

Everythig after a \# on a line is ignored!

Each action starts with the keyword \texttt{output}: the default
values are set (there are defauts for everything, but the data file
number). The activity is started with the key word \texttt{end}.
Everything in between is changing the default values. 


Several measurements can be merged together on 2 levels: 
- if all experimental parameters are identical one can just add
the data sets on the $I(t,z)$ level by giving several 
\texttt{input} or \texttt{direct} lines in the parameter file.

- to join several measurements at various $\omega$ values to get
one $R(q_z)$ curve one can use the following lines in the
amorreducer:

\texttt{output  | } \\
\texttt{grid    | } \\
\texttt{merge   | } 





\end{document}
